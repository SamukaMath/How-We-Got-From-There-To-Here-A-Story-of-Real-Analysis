\usepackage{geometry}
%% Some aspects of the preamble are conditional,
%% the LaTeX engine is one such determinant
\usepackage{ifthen}
%% etoolbox has a variety of modern conveniences
\usepackage{etoolbox}
\usepackage{ifxetex,ifluatex}
%% Raster graphics inclusion
\usepackage{graphicx}
%% Color support, xcolor package
%% Always loaded, for: add/delete text, author tools
%% Here, since tcolorbox loads tikz, and tikz loads xcolor
\PassOptionsToPackage{usenames,dvipsnames,svgnames,table}{xcolor}
\usepackage{xcolor}
%% begin: defined colors, via xcolor package, for styling
%% end: defined colors, via xcolor package, for styling
%% Colored boxes, and much more, though mostly styling
%% skins library provides "enhanced" skin, employing tikzpicture
%% boxes may be configured as "breakable" or "unbreakable"
%% "raster" controls grids of boxes, aka side-by-side
\usepackage{tcolorbox}
\tcbuselibrary{skins}
\tcbuselibrary{breakable}
\tcbuselibrary{raster}
%% We load some "stock" tcolorbox styles that we use a lot
%% Placement here is provisional, there will be some color work also
%% First, black on white, no border, transparent, but no assumption about titles
\tcbset{ bwminimalstyle/.style={size=minimal, boxrule=-0.3pt, frame empty,
		colback=white, colbacktitle=white, coltitle=black, opacityfill=0.0} }
%% Second, bold title, run-in to text/paragraph/heading
%% Space afterwards will be controlled by environment,
%% independent of constructions of the tcb title
%% Places \blocktitlefont onto many block titles
\tcbset{ runintitlestyle/.style={fonttitle=\blocktitlefont\upshape\bfseries, attach title to upper} }
%% Spacing prior to each exercise, anywhere
\tcbset{ exercisespacingstyle/.style={before skip={1.5ex plus 0.5ex}} }
%% Spacing prior to each block
\tcbset{ blockspacingstyle/.style={before skip={2.0ex plus 0.5ex}} }
%% xparse allows the construction of more robust commands,
%% this is a necessity for isolating styling and behavior
%% The tcolorbox library of the same name loads the base library
\tcbuselibrary{xparse}
%% The tcolorbox library loads TikZ, its calc package is generally useful,
%% and is necessary for some smaller documents that use partial tcolor boxes
%% See:  https://github.com/rbeezer/mathbook/issues/1624
\usetikzlibrary{calc}
%% Hyperref should be here, but likes to be loaded late
%%
%% Inline math delimiters, \(, \), need to be robust
%% 2016-01-31:  latexrelease.sty  supersedes  fixltx2e.sty
%% If  latexrelease.sty  exists, bugfix is in kernel
%% If not, bugfix is in  fixltx2e.sty
%% See:  https://tug.org/TUGboat/tb36-3/tb114ltnews22.pdf
%% and read "Fewer fragile commands" in distribution's  latexchanges.pdf
\IfFileExists{latexrelease.sty}{}{\usepackage{fixltx2e}}
%% shorter subnumbers in some side-by-side require manipulations
\usepackage{xstring}
%% Footnote counters and part/chapter counters are manipulated
%% April 2018:  chngcntr  commands now integrated into the kernel,
%% but circa 2018/2019 the package would still try to redefine them,
%% so we need to do the work of loading conditionally for old kernels.
%% From version 1.1a,  chngcntr  should detect defintions made by LaTeX kernel.
\ifdefined\counterwithin
\else
\usepackage{chngcntr}
\fi
%% Text height identically 9 inches, text width varies on point size
%% See Bringhurst 2.1.1 on measure for recommendations
%% 75 characters per line (count spaces, punctuation) is target
%% which is the upper limit of Bringhurst's recommendations
\geometry{letterpaper,total={340pt,9.0in}}
%% Custom Page Layout Adjustments (use latex.geometry)
%% This LaTeX file may be compiled with pdflatex, xelatex, or lualatex executables
%% LuaTeX is not explicitly supported, but we do accept additions from knowledgeable users
%% The conditional below provides  pdflatex  specific configuration last
%% begin: engine-specific capabilities
\ifthenelse{\boolean{xetex} \or \boolean{luatex}}{%
	%% begin: xelatex and lualatex-specific default configuration
	\ifxetex\usepackage{xltxtra}\fi
	%% realscripts is the only part of xltxtra relevant to lualatex 
	\ifluatex\usepackage{realscripts}\fi
	%% end:   xelatex and lualatex-specific default configuration
}{
	%% begin: pdflatex-specific default configuration
	%% We assume a PreTeXt XML source file may have Unicode characters
	%% and so we ask LaTeX to parse a UTF-8 encoded file
	%% This may work well for accented characters in Western language,
	%% but not with Greek, Asian languages, etc.
	%% When this is not good enough, switch to the  xelatex  engine
	%% where Unicode is better supported (encouraged, even)
	\usepackage[utf8]{inputenc}
	%% end: pdflatex-specific default configuration
}
%% end:   engine-specific capabilities
%%
%% Fonts.  Conditional on LaTex engine employed.
%% Default Text Font: The Latin Modern fonts are
%% "enhanced versions of the [original TeX] Computer Modern fonts."
%% We use them as the default text font for PreTeXt output.
%% Automatic Font Control
%% Portions of a document, are, or may, be affected by defined commands
%% These are perhaps more flexible when using  xelatex  rather than  pdflatex
%% The following definitions are meant to be re-defined in a style, using \renewcommand
%% They are scoped when employed (in a TeX group), and so should not be defined with an argument
\newcommand{\divisionfont}{\relax}
\newcommand{\blocktitlefont}{\relax}
\newcommand{\contentsfont}{\relax}
\newcommand{\pagefont}{\relax}
\newcommand{\tabularfont}{\relax}
\newcommand{\xreffont}{\relax}
\newcommand{\titlepagefont}{\relax}
%%
\ifthenelse{\boolean{xetex} \or \boolean{luatex}}{%
	%% begin: font setup and configuration for use with xelatex
	%% Generally, xelatex is necessary for non-Western fonts
	%% fontspec package provides extensive control of system fonts,
	%% meaning *.otf (OpenType), and apparently *.ttf (TrueType)
	%% that live *outside* your TeX/MF tree, and are controlled by your *system*
	%% (it is possible that a TeX distribution will place fonts in a system location)
	%%
	%% The fontspec package is the best vehicle for using different fonts in  xelatex
	%% So we load it always, no matter what a publisher or style might want
	%%
	\usepackage{fontspec}
	%%
	%% begin: xelatex main font ("font-xelatex-main" template)
	%% Latin Modern Roman is the default font for xelatex and so is loaded with a TU encoding
	%% *in the format* so we can't touch it, only perhaps adjust it later
	%% in one of two ways (then known by NFSS names such as "lmr")
	%% (1) via NFSS with font family names such as "lmr" and "lmss"
	%% (2) via fontspec with commands like \setmainfont{Latin Modern Roman}
	%% The latter requires the font to be known at the system-level by its font name,
	%% but will give access to OTF font features through optional arguments
	%% https://tex.stackexchange.com/questions/470008/
	%% where-and-how-does-fontspec-sty-specify-the-default-font-latin-modern-roman
	%% http://tex.stackexchange.com/questions/115321
	%% /how-to-optimize-latin-modern-font-with-xelatex
	%%
	%% end:   xelatex main font ("font-xelatex-main" template)
	%% begin: xelatex mono font ("font-xelatex-mono" template)
	%% (conditional on non-trivial uses being present in source)
	%% end:   xelatex mono font ("font-xelatex-mono" template)
	%% begin: xelatex font adjustments ("font-xelatex-style" template)
	%% end:   xelatex font adjustments ("font-xelatex-style" template)
	%%
	%% Extensive support for other languages
	\usepackage{polyglossia}
	%% Set main/default language based on pretext/@xml:lang value
	%% document language code is "en-US", US English
	%% usmax variant has extra hypenation
	\setmainlanguage[variant=usmax]{english}
	%% Enable secondary languages based on discovery of @xml:lang values
	%% Enable fonts/scripts based on discovery of @xml:lang values
	%% Western languages should be ably covered by Latin Modern Roman
	%% end:   font setup and configuration for use with xelatex
}{%
	%% begin: font setup and configuration for use with pdflatex
	%% begin: pdflatex main font ("font-pdflatex-main" template)
	\usepackage{lmodern}
	\usepackage[T1]{fontenc}
	%% end:   pdflatex main font ("font-pdflatex-main" template)
	%% begin: pdflatex mono font ("font-pdflatex-mono" template)
	%% (conditional on non-trivial uses being present in source)
	%% end:   pdflatex mono font ("font-pdflatex-mono" template)
	%% begin: pdflatex font adjustments ("font-pdflatex-style" template)
	%% end:   pdflatex font adjustments ("font-pdflatex-style" template)
	%% end:   font setup and configuration for use with pdflatex
}
%% Micromanage spacing, etc.  The named "microtype-options"
%% template may be employed to fine-tune package behavior
\usepackage{microtype}
%% Symbols, align environment, commutative diagrams, bracket-matrix
\usepackage{amsmath}
\usepackage{amscd}
\usepackage{amssymb}
\usepackage[brazil]{babel}
%% allow page breaks within display mathematics anywhere
%% level 4 is maximally permissive
%% this is exactly the opposite of AMSmath package philosophy
%% there are per-display, and per-equation options to control this
%% split, aligned, gathered, and alignedat are not affected
\allowdisplaybreaks[4]
%% allow more columns to a matrix
%% can make this even bigger by overriding with  latex.preamble.late  processing option
\setcounter{MaxMatrixCols}{30}
%%
%%
%% Division Titles, and Page Headers/Footers
%% titlesec package, loading "titleps" package cooperatively
%% See code comments about the necessity and purpose of "explicit" option.
%% The "newparttoc" option causes a consistent entry for parts in the ToC 
%% file, but it is only effective if there is a \titleformat for \part.
%% "pagestyles" loads the  titleps  package cooperatively.
\usepackage[explicit, newparttoc, pagestyles]{titlesec}
%% The companion titletoc package for the ToC.
\usepackage{titletoc}
%% Fixes a bug with transition from chapters to appendices in a "book"
%% See generating XSL code for more details about necessity
\newtitlemark{\chaptertitlename}
%% begin: customizations of page styles via the modal "titleps-style" template
%% Designed to use commands from the LaTeX "titleps" package
%% Plain pages should have the same font for page numbers
\renewpagestyle{plain}{%
	\setfoot{}{\pagefont\thepage}{}%
}%
%% Single pages as in default LaTeX
\renewpagestyle{headings}{%
	\sethead{\pagefont\slshape\MakeUppercase{\ifthechapter{\chaptertitlename\space\thechapter.\space}{}\chaptertitle}}{}{\pagefont\thepage}%
}%
\pagestyle{headings}
%% end: customizations of page styles via the modal "titleps-style" template
%%
%% Create globally-available macros to be provided for style writers
%% These are redefined for each occurence of each division
\newcommand{\divisionnameptx}{\relax}%
\newcommand{\titleptx}{\relax}%
\newcommand{\subtitleptx}{\relax}%
\newcommand{\shortitleptx}{\relax}%
\newcommand{\authorsptx}{\relax}%
\newcommand{\epigraphptx}{\relax}%
%% Create environments for possible occurences of each division
%% Environment for a PTX "acknowledgement" at the level of a LaTeX "chapter"
\NewDocumentEnvironment{acknowledgement}{mmmmmm}
{%
	\renewcommand{\divisionnameptx}{Acknowledgements}%
	\renewcommand{\titleptx}{#1}%
	\renewcommand{\subtitleptx}{#2}%
	\renewcommand{\shortitleptx}{#3}%
	\renewcommand{\authorsptx}{#4}%
	\renewcommand{\epigraphptx}{#5}%
	\chapter*{#1}%
	\addcontentsline{toc}{chapter}{#3}
	\label{#6}%
}{}%
%% Environment for a PTX "chapter" at the level of a LaTeX "chapter"
\NewDocumentEnvironment{chapterptx}{mmmmmm}
{%
	\renewcommand{\divisionnameptx}{Capítulo}%
	\renewcommand{\titleptx}{#1}%
	\renewcommand{\subtitleptx}{#2}%
	\renewcommand{\shortitleptx}{#3}%
	\renewcommand{\authorsptx}{#4}%
	\renewcommand{\epigraphptx}{#5}%
	\chapter[{#3}]{#1}%
	\label{#6}%
}{}%
%% Environment for a PTX "section" at the level of a LaTeX "section"
\NewDocumentEnvironment{sectionptx}{mmmmmm}
{%
	\renewcommand{\divisionnameptx}{Seção}%
	\renewcommand{\titleptx}{#1}%
	\renewcommand{\subtitleptx}{#2}%
	\renewcommand{\shortitleptx}{#3}%
	\renewcommand{\authorsptx}{#4}%
	\renewcommand{\epigraphptx}{#5}%
	\section[{#3}]{#1}%
	\label{#6}%
}{}%
%% Environment for a PTX "part" at the level of a LaTeX "part"
\NewDocumentEnvironment{partptx}{mmmmmm}
{%
	\renewcommand{\divisionnameptx}{Part}%
	\renewcommand{\titleptx}{#1}%
	\renewcommand{\subtitleptx}{#2}%
	\renewcommand{\shortitleptx}{#3}%
	\renewcommand{\authorsptx}{#4}%
	\renewcommand{\epigraphptx}{#5}%
	\part[{#3}]{#1}%
	\label{#6}%
}{}%
%% Environment for a PTX "subsection" at the level of a LaTeX "subsection"
\NewDocumentEnvironment{subsectionptx}{mmmmmm}
{%
	\renewcommand{\divisionnameptx}{Subseção}%
	\renewcommand{\titleptx}{#1}%
	\renewcommand{\subtitleptx}{#2}%
	\renewcommand{\shortitleptx}{#3}%
	\renewcommand{\authorsptx}{#4}%
	\renewcommand{\epigraphptx}{#5}%
	\subsection[{#3}]{#1}%
	\label{#6}%
}{}%
%% Environment for a PTX "references" at the level of a LaTeX "chapter"
\NewDocumentEnvironment{references-chapter}{mmmmmm}
{%
	\renewcommand{\divisionnameptx}{Referências}%
	\renewcommand{\titleptx}{#1}%
	\renewcommand{\subtitleptx}{#2}%
	\renewcommand{\shortitleptx}{#3}%
	\renewcommand{\authorsptx}{#4}%
	\renewcommand{\epigraphptx}{#5}%
	\chapter[{#3}]{#1}%
	\label{#6}%
}{}%
%% Environment for a PTX "references" at the level of a LaTeX "chapter"
\NewDocumentEnvironment{references-chapter-numberless}{mmmmmm}
{%
	\renewcommand{\divisionnameptx}{Referências}%
	\renewcommand{\titleptx}{#1}%
	\renewcommand{\subtitleptx}{#2}%
	\renewcommand{\shortitleptx}{#3}%
	\renewcommand{\authorsptx}{#4}%
	\renewcommand{\epigraphptx}{#5}%
	\chapter*{#1}%
	\addcontentsline{toc}{chapter}{#3}
	\label{#6}%
}{}%
%% Environment for a PTX "index" at the level of a LaTeX "chapter"
\NewDocumentEnvironment{indexptx}{mmmmmm}
{%
	\renewcommand{\divisionnameptx}{Índices}%
	\renewcommand{\titleptx}{#1}%
	\renewcommand{\subtitleptx}{#2}%
	\renewcommand{\shortitleptx}{#3}%
	\renewcommand{\authorsptx}{#4}%
	\renewcommand{\epigraphptx}{#5}%
	\chapter*{#1}%
	\addcontentsline{toc}{chapter}{#3}
	\label{#6}%
}{}%
%%
%% Styles for six traditional LaTeX divisions
\titleformat{\part}[display]
{\divisionfont\Huge\bfseries\centering}{\divisionnameptx\space\thepart}{30pt}{\Huge#1}
[{\Large\centering\authorsptx}]
\titleformat{\chapter}[display]
{\divisionfont\huge\bfseries}{\divisionnameptx\space\thechapter}{20pt}{\Huge#1}
[{\Large\authorsptx}]
\titleformat{name=\chapter,numberless}[display]
{\divisionfont\huge\bfseries}{}{0pt}{#1}
[{\Large\authorsptx}]
\titlespacing*{\chapter}{0pt}{50pt}{40pt}
\titleformat{\section}[hang]
{\divisionfont\Large\bfseries}{\thesection}{1ex}{#1}
[{\large\authorsptx}]
\titleformat{name=\section,numberless}[block]
{\divisionfont\Large\bfseries}{}{0pt}{#1}
[{\large\authorsptx}]
\titlespacing*{\section}{0pt}{3.5ex plus 1ex minus .2ex}{2.3ex plus .2ex}
\titleformat{\subsection}[hang]
{\divisionfont\large\bfseries}{\thesubsection}{1ex}{#1}
[{\normalsize\authorsptx}]
\titleformat{name=\subsection,numberless}[block]
{\divisionfont\large\bfseries}{}{0pt}{#1}
[{\normalsize\authorsptx}]
\titlespacing*{\subsection}{0pt}{3.25ex plus 1ex minus .2ex}{1.5ex plus .2ex}
\titleformat{\subsubsection}[hang]
{\divisionfont\normalsize\bfseries}{\thesubsubsection}{1em}{#1}
[{\small\authorsptx}]
\titleformat{name=\subsubsection,numberless}[block]
{\divisionfont\normalsize\bfseries}{}{0pt}{#1}
[{\normalsize\authorsptx}]
\titlespacing*{\subsubsection}{0pt}{3.25ex plus 1ex minus .2ex}{1.5ex plus .2ex}
\titleformat{\paragraph}[hang]
{\divisionfont\normalsize\bfseries}{\theparagraph}{1em}{#1}
[{\small\authorsptx}]
\titleformat{name=\paragraph,numberless}[block]
{\divisionfont\normalsize\bfseries}{}{0pt}{#1}
[{\normalsize\authorsptx}]
\titlespacing*{\paragraph}{0pt}{3.25ex plus 1ex minus .2ex}{1.5em}
%%
%% Styles for five traditional LaTeX divisions
\titlecontents{part}%
[0pt]{\contentsmargin{0em}\addvspace{1pc}\contentsfont\bfseries}%
{\Large\thecontentslabel\enspace}{\Large}%
{}%
[\addvspace{.5pc}]%
\titlecontents{chapter}%
[0pt]{\contentsmargin{0em}\addvspace{1pc}\contentsfont\bfseries}%
{\large\thecontentslabel\enspace}{\large}%
{\hfill\bfseries\thecontentspage}%
[\addvspace{.5pc}]%
\dottedcontents{section}[3.8em]{\contentsfont}{2.3em}{1pc}%
\dottedcontents{subsection}[6.1em]{\contentsfont}{3.2em}{1pc}%
\dottedcontents{subsubsection}[9.3em]{\contentsfont}{4.3em}{1pc}%
%%
%% Begin: Semantic Macros
%% To preserve meaning in a LaTeX file
%%
%% \mono macro for content of "c", "cd", "tag", etc elements
%% Also used automatically in other constructions
%% Simply an alias for \texttt
%% Always defined, even if there is no need, or if a specific tt font is not loaded
\newcommand{\mono}[1]{\texttt{#1}}
%%
%% Following semantic macros are only defined here if their
%% use is required only in this specific document
%%
%% Used for warnings, typically bold and italic
\newcommand{\alert}[1]{\textbf{\textit{#1}}}
%% Used for inline definitions of terms
\newcommand{\terminology}[1]{\textbf{#1}}
%% Style of a title on a list item, for ordered and unordered lists
%% Also "task" of exercise, PROJECT-LIKE, EXAMPLE-LIKE
\newcommand{\lititle}[1]{{\slshape#1}}
%% End: Semantic Macros
%% Localize LaTeX supplied names (possibly none)
\renewcommand*{\partname}{Part}
\renewcommand*{\chaptername}{Chapter}
%% Equation Numbering
%% Controlled by  numbering.equations.level  processing parameter
%% No adjustment here implies document-wide numbering
\numberwithin{equation}{part}
%% "tcolorbox" environment for a single image, occupying entire \linewidth
%% arguments are left-margin, width, right-margin, as multiples of
%% \linewidth, and are guaranteed to be positive and sum to 1.0
\tcbset{ imagestyle/.style={bwminimalstyle} }
\NewTColorBox{image}{mmm}{imagestyle,left skip=#1\linewidth,width=#2\linewidth}
%% Footnote Numbering
%% Specified by numbering.footnotes.level
%% Undo counter reset by chapter for a book
\counterwithout{footnote}{chapter}
%% Global numbering, since numbering.footnotes.level = 0
%% QR Code Support
%% Videos and other interactives
\usepackage{qrcode}
\newlength{\qrsize}
\newlength{\previewwidth}
%% tcolorbox styles for interactive previews
%% changing size= and/or colback can aid in debugging
\tcbset{ previewstyle/.style={bwminimalstyle, halign=center} }
\tcbset{ qrstyle/.style={bwminimalstyle, hbox} }
\tcbset{ captionstyle/.style={bwminimalstyle, left=1em, width=\linewidth} }
%% Generic red play button (from SVG)
%% tikz package should be loaded by now
\definecolor{playred}{RGB}{230,33,23}
\newcommand{\genericpreview}{
	\begin{tikzpicture}[y=0.80pt, x=0.80pt, yscale=-1.000000, xscale=1.000000, inner sep=0pt, outer sep=0pt]
		\path[fill=playred] (94.9800,28.8400) .. controls (94.9800,28.8400) and
		(94.0400,22.2400) .. (91.1700,19.3400) .. controls (87.5300,15.5300) and
		(83.4500,15.5100) .. (81.5800,15.2900) .. controls (68.1800,14.3200) and
		(48.0600,14.4400) .. (48.0600,14.4400) .. controls (48.0600,14.4400) and
		(27.9400,14.3200) .. (14.5400,15.2900) .. controls (12.6700,15.5100) and
		(8.5900,15.5300) .. (4.9500,19.3400) .. controls (2.0800,22.2400) and
		(1.1400,28.8400) .. (1.1400,28.8400) .. controls (1.1400,28.8400) and
		(0.1800,36.5800) .. (0.0000,44.3300) -- (0.0000,51.5900) .. controls
		(0.1800,59.3400) and (1.1400,67.0800) .. (1.1400,67.0800) .. controls
		(1.1400,67.0800) and (2.0700,73.6800) .. (4.9500,76.5800) .. controls
		(8.5900,80.3900) and (13.3800,80.2700) .. (15.5100,80.6700) .. controls
		(23.0400,81.3900) and (47.2100,81.5600) .. (48.0500,81.5700) .. controls
		(48.0600,81.5700) and (68.1900,81.6000) .. (81.5900,80.6300) .. controls
		(83.4600,80.4100) and (87.5400,80.3900) .. (91.1800,76.5800) .. controls
		(94.0500,73.6800) and (94.9900,67.0800) .. (94.9900,67.0800) .. controls
		(94.9900,67.0800) and (95.9500,59.3300) .. (96.0100,51.5900) --
		(96.0100,44.3300) .. controls (95.9400,36.5800) and (94.9800,28.8400) ..
		(94.9800,28.8400) -- cycle(38.2800,61.4100) -- (38.2800,34.4100) --
		(64.0200,47.9100) -- (38.2800,61.4100) -- cycle;
	\end{tikzpicture}
}
%% More flexible list management, esp. for references
%% But also for specifying labels (i.e. custom order) on nested lists
\usepackage{enumitem}
%% Lists of references in their own section, maximum depth 1
\newlist{referencelist}{description}{4}
\setlist[referencelist]{leftmargin=!,labelwidth=!,labelsep=0ex,itemsep=1.0ex,topsep=1.0ex,partopsep=0pt,parsep=0pt}
%% Description lists as tcolorbox sidebyside
%% "dli" short for "description list item"
\newlength{\dlititlewidth}
\newlength{\dlimaxnarrowtitle}\setlength{\dlimaxnarrowtitle}{11ex}
\newlength{\dlimaxmediumtitle}\setlength{\dlimaxmediumtitle}{18ex}
\tcbset{ dlistyle/.style={sidebyside, sidebyside align=top seam, lower separated=false, bwminimalstyle, bottomtitle=0.75ex, after skip=1.5ex, boxsep=0pt, left=0pt, right=0pt, top=0pt, bottom=0pt} }
\tcbset{ dlinarrowstyle/.style={dlistyle, lefthand width=\dlimaxnarrowtitle, sidebyside gap=1ex, halign=flush left, righttitle=10ex} }
\tcbset{ dlimediumstyle/.style={dlistyle, lefthand width=\dlimaxmediumtitle, sidebyside gap=4ex, halign=flush right} }
\NewDocumentEnvironment{descriptionlist}{}{\par\vspace*{1.5ex}}{\par\vspace*{1.5ex}}%
%% begin enviroment has an if/then to open the tcolorbox
\NewDocumentEnvironment{dlinarrow}{mm}{%
	\settowidth{\dlititlewidth}{{\textbf{#1}}}%
	\ifthenelse{\dlititlewidth > \dlimaxnarrowtitle}%
	{\begin{tcolorbox}[title={\textbf{#1}}, phantom={\hypertarget{#2}{}}, dlinarrowstyle]\tcblower}%
		{\begin{tcolorbox}[dlinarrowstyle, phantom={\hypertarget{#2}{}}]\textbf{#1}\tcblower}%
		}%
		{\end{tcolorbox}}%
	%% medium option is simpler
	\NewDocumentEnvironment{dlimedium}{mm}%
	{\begin{tcolorbox}[dlimediumstyle, phantom={\hypertarget{#2}{}}]\textbf{#1}\tcblower}%
		{\end{tcolorbox}}%
	%% Support for index creation
	%% imakeidx package does not require extra pass (as with makeidx)
	%% Title of the "Index" section set via a keyword
	%% Language support for the "see" and "see also" phrases
	\usepackage{imakeidx}
	\makeindex[title=Índices, intoc=true]
	\renewcommand{\seename}{See}
	\renewcommand{\alsoname}{See also}
	%% hyperref driver does not need to be specified, it will be detected
	%% Footnote marks in tcolorbox have broken linking under
	%% hyperref, so it is necessary to turn off all linking
	%% It *must* be given as a package option, not with \hypersetup
	\usepackage[hyperfootnotes=false]{hyperref}
	%% configure hyperref's  \href{}{}  and  \nolinkurl  to match listings' inline verbatim
	\renewcommand\UrlFont{\small\ttfamily}
	%% Hyperlinking active in electronic PDFs, all links without surrounding boxes and blue
	\hypersetup{colorlinks=true,linkcolor=blue,citecolor=blue,filecolor=blue,urlcolor=blue}
	\hypersetup{pdftitle={How We Got From There To Here: A Story of Real Analysis}}
	%% If you manually remove hyperref, leave in this next command
	%% This will allow LaTeX compilation, employing this no-op command
	\providecommand\phantomsection{}
	%% Division Numbering: Chapters, Sections, Subsections, etc
	%% Division numbers may be turned off at some level ("depth")
	%% A section *always* has depth 1, contrary to us counting from the document root
	%% The latex default is 3.  If a larger number is present here, then
	%% removing this command may make some cross-references ambiguous
	%% The precursor variable $numbering-maxlevel is checked for consistency in the common XSL file
	\setcounter{secnumdepth}{3}
	%%
	%% AMS "proof" environment is no longer used, but we leave previously
	%% implemented \qedhere in place, should the LaTeX be recycled
	\newcommand{\qedhere}{\relax}
	%%
	%% A faux tcolorbox whose only purpose is to provide common numbering
	%% facilities for most blocks (possibly not projects, 2D displays)
	%% Controlled by  numbering.theorems.level  processing parameter
	\newtcolorbox[auto counter, number within=chapter]{block}{}
	%%
	%% This document is set to number PROJECT-LIKE on a separate numbering scheme
	%% So, a faux tcolorbox whose only purpose is to provide this numbering
	%% Controlled by  numbering.projects.level  processing parameter
	\newtcolorbox[auto counter]{project-distinct}{}
	%% A faux tcolorbox whose only purpose is to provide common numbering
	%% facilities for 2D displays which are subnumbered as part of a "sidebyside"
	\makeatletter
	\newtcolorbox[auto counter, number within=tcb@cnt@block, number freestyle={\noexpand\thetcb@cnt@block(\noexpand\alph{\tcbcounter})}]{subdisplay}{}
	\makeatother
	%%
	%% tcolorbox, with styles, for THEOREM-LIKE
	%%
	%% theorem: fairly simple numbered block/structure
	\tcbset{ theoremstyle/.style={bwminimalstyle, runintitlestyle, blockspacingstyle, after title={\space}, } }
	\newtcolorbox[use counter from=block]{theorem}[3]{title={{Theorem~\thetcbcounter\notblank{#1#2}{\space}{}\notblank{#1}{\space#1}{}\notblank{#2}{\space(#2)}{}}}, phantomlabel={#3}, breakable, parbox=false, after={\par}, fontupper=\itshape, theoremstyle, }
	%% lemma: fairly simple numbered block/structure
	\tcbset{ lemmastyle/.style={bwminimalstyle, runintitlestyle, blockspacingstyle, after title={\space}, } }
	\newtcolorbox[use counter from=block]{lemma}[3]{title={{Lemma~\thetcbcounter\notblank{#1#2}{\space}{}\notblank{#1}{\space#1}{}\notblank{#2}{\space(#2)}{}}}, phantomlabel={#3}, breakable, parbox=false, after={\par}, fontupper=\itshape, lemmastyle, }
	%% corollary: fairly simple numbered block/structure
	\tcbset{ corollarystyle/.style={bwminimalstyle, runintitlestyle, blockspacingstyle, after title={\space}, } }
	\newtcolorbox[use counter from=block]{corollary}[3]{title={{Corollary~\thetcbcounter\notblank{#1#2}{\space}{}\notblank{#1}{\space#1}{}\notblank{#2}{\space(#2)}{}}}, phantomlabel={#3}, breakable, parbox=false, after={\par}, fontupper=\itshape, corollarystyle, }
	%%
	%% tcolorbox, with styles, for AXIOM-LIKE
	%%
	%% principle: fairly simple numbered block/structure
	\tcbset{ principlestyle/.style={bwminimalstyle, runintitlestyle, blockspacingstyle, after title={\space}, } }
	\newtcolorbox[use counter from=block]{principle}[3]{title={{Principle~\thetcbcounter\notblank{#1#2}{\space}{}\notblank{#1}{\space#1}{}\notblank{#2}{\space(#2)}{}}}, phantomlabel={#3}, breakable, parbox=false, after={\par}, fontupper=\itshape, principlestyle, }
	%% conjecture: fairly simple numbered block/structure
	\tcbset{ conjecturestyle/.style={bwminimalstyle, runintitlestyle, blockspacingstyle, after title={\space}, } }
	\newtcolorbox[use counter from=block]{conjecture}[3]{title={{Conjecture~\thetcbcounter\notblank{#1#2}{\space}{}\notblank{#1}{\space#1}{}\notblank{#2}{\space(#2)}{}}}, phantomlabel={#3}, breakable, parbox=false, after={\par}, fontupper=\itshape, conjecturestyle, }
	%% axiom: fairly simple numbered block/structure
	\tcbset{ axiomstyle/.style={bwminimalstyle, runintitlestyle, blockspacingstyle, after title={\space}, } }
	\newtcolorbox[use counter from=block]{axiom}[3]{title={{Axiom~\thetcbcounter\notblank{#1#2}{\space}{}\notblank{#1}{\space#1}{}\notblank{#2}{\space(#2)}{}}}, phantomlabel={#3}, breakable, parbox=false, after={\par}, fontupper=\itshape, axiomstyle, }
	%%
	%% tcolorbox, with styles, for DEFINITION-LIKE
	%%
	%% definition: fairly simple numbered block/structure
	\tcbset{ definitionstyle/.style={bwminimalstyle, runintitlestyle, blockspacingstyle, after title={\space}, after upper={\space\space\hspace*{\stretch{1}}\(\lozenge\)}, } }
	\newtcolorbox[use counter from=block]{definition}[2]{title={{Definition~\thetcbcounter\notblank{#1}{\space\space#1}{}}}, phantomlabel={#2}, breakable, parbox=false, after={\par}, definitionstyle, }
	%%
	%% tcolorbox, with styles, for EXAMPLE-LIKE
	%%
	%% problem: fairly simple numbered block/structure
	\tcbset{ problemstyle/.style={bwminimalstyle, runintitlestyle, blockspacingstyle, after title={\space}, after upper={\space\space\hspace*{\stretch{1}}\(\square\)}, } }
	\newtcolorbox[use counter from=block]{problem}[2]{title={{Problem~\thetcbcounter\notblank{#1}{\space\space#1}{}}}, phantomlabel={#2}, breakable, parbox=false, after={\par}, problemstyle, }
	%% example: fairly simple numbered block/structure
	\tcbset{ examplestyle/.style={bwminimalstyle, runintitlestyle, blockspacingstyle, after title={\space}, after upper={\space\space\hspace*{\stretch{1}}\(\square\)}, } }
	\newtcolorbox[use counter from=block]{example}[2]{title={{Example~\thetcbcounter\notblank{#1}{\space\space#1}{}}}, phantomlabel={#2}, breakable, parbox=false, after={\par}, examplestyle, }
	%%
	%% tcolorbox, with styles, for ASIDE-LIKE
	%%
	%% aside: fairly simple un-numbered block/structure
	\tcbset{ asidestyle/.style={bwminimalstyle, runintitlestyle, blockspacingstyle, after title={\space}, } }
	\newtcolorbox{aside}[2]{title={\notblank{#1}{#1}{}}, phantomlabel={#2}, breakable, parbox=false, asidestyle}
	%% historical: fairly simple un-numbered block/structure
	\tcbset{ historicalstyle/.style={bwminimalstyle, runintitlestyle, blockspacingstyle, after title={\space}, } }
	\newtcolorbox{historical}[2]{title={\notblank{#1}{#1}{}}, phantomlabel={#2}, breakable, parbox=false, historicalstyle}
	%%
	%% tcolorbox, with styles, for FIGURE-LIKE
	%%
	%% figureptx: 2-D display structure
	\tcbset{ figureptxstyle/.style={bwminimalstyle, middle=1ex, blockspacingstyle, fontlower=\blocktitlefont} }
	\newtcolorbox[use counter from=block]{figureptx}[3]{lower separated=false, before lower={{\textbf{Figure~\thetcbcounter}\space#1}}, phantomlabel={#2}, unbreakable, parbox=false, figureptxstyle, }
	%%
	%% xparse environments for introductions and conclusions of divisions
	%%
	%% introduction: in a structured division
	\NewDocumentEnvironment{introduction}{m}
	{\notblank{#1}{\noindent\textbf{#1}\space}{}}{\par\medskip}
	%%
	%% tcolorbox, with styles, for miscellaneous environments
	%%
	%% proof: title is a replacement
	\tcbset{ proofstyle/.style={bwminimalstyle, fonttitle=\blocktitlefont\itshape, attach title to upper, after title={\space}, after upper={\space\space\hspace*{\stretch{1}}\(\blacksquare\)},
	} }
	\newtcolorbox{proof}[2]{title={\notblank{#1}{#1}{Proof.}}, phantom={\hypertarget{#2}{}}, breakable, parbox=false, after={\par}, proofstyle }
	%% Graphics Preamble Entries
	
	%% If tikz has been loaded, replace ampersand with \amp macro
	%% tcolorbox styles for sidebyside layout
	\tcbset{ sbsstyle/.style={raster before skip=2.0ex, raster equal height=rows, raster force size=false} }
	\tcbset{ sbspanelstyle/.style={bwminimalstyle, fonttitle=\blocktitlefont} }
	%% Enviroments for side-by-side and components
	%% Necessary to use \NewTColorBox for boxes of the panels
	%% "newfloat" environment to squash page-breaks within a single sidebyside
	%% "xparse" environment for entire sidebyside
	\NewDocumentEnvironment{sidebyside}{mmmm}
	{\begin{tcbraster}
			[sbsstyle,raster columns=#1,
			raster left skip=#2\linewidth,raster right skip=#3\linewidth,raster column skip=#4\linewidth]}
		{\end{tcbraster}}
	%% "tcolorbox" environment for a panel of sidebyside
	\NewTColorBox{sbspanel}{mO{top}}{sbspanelstyle,width=#1\linewidth,valign=#2}
	%% extpfeil package for certain extensible arrows,
	%% as also provided by MathJax extension of the same name
	%% NB: this package loads mtools, which loads calc, which redefines
	%%     \setlength, so it can be removed if it seems to be in the 
	%%     way and your math does not use:
	%%     
	%%     \xtwoheadrightarrow, \xtwoheadleftarrow, \xmapsto, \xlongequal, \xtofrom
	%%     
	%%     we have had to be extra careful with variable thickness
	%%     lines in tables, and so also load this package late
	\usepackage{extpfeil}
	%% Custom Preamble Entries, late (use latex.preamble.late)
	%% Begin: Author-provided packages
	%% (From  docinfo/latex-preamble/package  elements)
	%% End: Author-provided packages
	%% Begin: Author-provided macros
	%% (From  docinfo/macros  element)
	%% Plus three from MBX for XML characters
	\newcommand{\imp}{\ \Rightarrow\ }
	\newcommand{\dx}[1]{\,{\rm d}#1}
	
	
	\newcommand{\dfdx}[2]{\frac{\text{d}{#1}}{\text{d}{#2}}}
	\newcommand{\abs}[1]{\left|#1\right|}
	\def\limit#1#2#3{{\displaystyle\lim_{#1\rightarrow #2}#3}}
	\newcommand{\eps}{\varepsilon}
	\newcommand{\unif}{\stackrel{unif}{\longrightarrow}}
	\newcommand{\ptwise}{\stackrel{ptwise}{\longrightarrow}}
	\newcommand{\CC}{\mathbb {C}}
	\newcommand{\DD}{\mathbb {D}}
	\newcommand{\RR}{\mathbb {R}}
	\newcommand{\QQ}{\mathbb {Q}}
	\newcommand{\NN}{\mathbb {N}}
	\newcommand{\ZZ}{\mathbb {Z}}
	\newcommand{\lt}{<}
	\newcommand{\gt}{>}
	\newcommand{\amp}{&}