\typeout{************************************************}
\typeout{Referências  Bibliograficas}
\typeout{************************************************}
%
\begin{references-chapter-numberless}{Referências}{}{Referências}{}{}{g:Referências:idp408}
	%% If this is a top-level references
	%%   you can replace with "thebibliography" environment
	\begin{referencelist}
		\bibitem[1]{x:biblio:bradley09__cauch_cours}\hypertarget{x:biblio:bradley09__cauch_cours}{}Robert E. Bradley and C. Edward Sandifer. \emph{\textit{Cauchy's Cours d'analyse}: An Annotated Translation}. Sources and Studies in the History of Mathematics and Physical Sciences. Springer, 2009.
		\bibitem[2]{x:biblio:dunham90__journ_throug_genius}\hypertarget{x:biblio:dunham90__journ_throug_genius}{}William Dunham. \emph{Journey Through Genius}. Penguin Books, 1990.
		\bibitem[3]{x:biblio:franks10__cantor_other_proof_rr_uncoun}\hypertarget{x:biblio:franks10__cantor_other_proof_rr_uncoun}{}John Franks. Cantor's other proofs that \(\RR\) is uncountable. \emph{Mathematics Magazine}, 83(4):283-{}-{}289, October 2010.
		\bibitem[4]{x:biblio:grabiner81__origin_cauch_rigor_calculy}\hypertarget{x:biblio:grabiner81__origin_cauch_rigor_calculy}{}Judith Grabiner. \emph{The Origins of Cauchy's Rigorous Calculus}. MIT Press, Cambridge MA, 1981.
		\bibitem[5]{x:biblio:hawking05__god_creat_integ}\hypertarget{x:biblio:hawking05__god_creat_integ}{}Stephen Hawking, editor. \emph{God Created the Integers: The Mathematical Breakthroughs that Changed History}. Running Press, Philadelphia, London, 2005.
		\bibitem[6]{x:biblio:jahnke03__histor_analy}\hypertarget{x:biblio:jahnke03__histor_analy}{}H. Jahnke, editor. \emph{A History of Analysis}. AMS Publications, Providence RI, 2003.
		\bibitem[7]{x:biblio:landau66__found_analy}\hypertarget{x:biblio:landau66__found_analy}{}Edmund Landau. \emph{Foundations of Analysis}. Chelsea Publishing Company, New York, NY, 1966. Translated by F. Steinhardt, Columbia University.
		\bibitem[8]{x:biblio:levenson09__newton_count}\hypertarget{x:biblio:levenson09__newton_count}{}Thomas Levenson. \emph{Newton and the Counterfeiter}. Houghton Mifflin Harcourt, 2009.
		\bibitem[9]{x:biblio:netz07__archim_codex}\hypertarget{x:biblio:netz07__archim_codex}{}Reviel Netz and William Noel. \emph{The Archimedes Codex}. Da Capo Press, 2007.
		\bibitem[10]{x:biblio:newton45__sir_isaac_two_treat_quadr}\hypertarget{x:biblio:newton45__sir_isaac_two_treat_quadr}{}Isaac Newton. \emph{Sir Isaac Newton's Two Treatises of the Quadrature of Curves and Analysis by Equation of an Infinite Number of Terms, Explained}. Society for the Encouragement of Learning, 1745. Translated from Latin by John Stewart, A. M. Professor of Mathematicks in the Marishal College and University of Aberdeen.
		\bibitem[11]{x:biblio:Bernoulli_bio_mactutor}\hypertarget{x:biblio:Bernoulli_bio_mactutor}{}J. J. O'Connor and E. F. Robertson. \emph{The Brachistochrone Problem}. \emph{http:\slash{}\slash{}www-gap.dcs.st-and.ac.uk", MacTutor History of Mathematics archive}
		\bibitem[12]{x:biblio:robinson74__non_stand_analy}\hypertarget{x:biblio:robinson74__non_stand_analy}{}Abraham Robinson. \emph{Non-standard analysis}. North-Holland Pub. Co., 1974.
		\bibitem[13]{x:biblio:russo96__forgot_revol}\hypertarget{x:biblio:russo96__forgot_revol}{}Lucio Russo, translated by Silvio Levy. \emph{The Forgotten Revolution: How Science Was Born in 300 BC and Why It Had to Be Reborn}. Springer, 1996.
		\bibitem[14]{x:biblio:sandifer07__early_mathem_leonar_euler}\hypertarget{x:biblio:sandifer07__early_mathem_leonar_euler}{}C. Edward Sandifer. \emph{The Early Mathematics of Leonard Euler}. Spectrum, 2007. ISBN 10: 0883855593, ISBN 13: 978-0883855591.
		\bibitem[15]{x:biblio:struik69__sourc_book_mathem}\hypertarget{x:biblio:struik69__sourc_book_mathem}{}Dirk Struik, editor. \emph{Source Book in Mathematics, 1200-1800}. Harvard University Press, Cambridge, MA, 1969.
		\bibitem[16]{x:biblio:thurston56__number_system}\hypertarget{x:biblio:thurston56__number_system}{}H. A. Thurston. \emph{The Number System}. Blackie and Son Limited, London, Glassgow, 1956.
	\end{referencelist}
\end{references-chapter-numberless}
%